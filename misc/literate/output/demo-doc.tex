\documentclass{article}

% Defining colors by names
\usepackage{xcolor}
% Verbatim enviroment
\usepackage{fancyvrb}
% Verbatim enviroment for unformatted source code
\usepackage{listings}
% Better font for backslash
\usepackage[T1]{fontenc}
\usepackage{hyperref}
% Providing more features than usual tabular
\usepackage{longtable}

% NOTE: 
% Remember to use big letters for color codes
% Reference at http://tex.stackexchange.com/questions/18008/use-css-style-color-specifications-in-xcolor

% Identifiers
\newcommand{\id}[1]{\textcolor[HTML]{000000}{#1}}

% Strings
\newcommand{\str}[1]{\textcolor[HTML]{A31515}{#1}}

% Keywords
\newcommand{\kwd}[1]{\textcolor[HTML]{0000FF}{#1}}

% Comments
\newcommand{\com}[1]{\textcolor[HTML]{008000}{#1}}

% Operators
\newcommand{\ops}[1]{\textcolor[HTML]{000000}{#1}}

% Numbers
\newcommand{\num}[1]{\textcolor[HTML]{000000}{#1}}

% Line number
\newcommand{\lines}[1]{\textcolor[HTML]{96C2CD}{#1}}

% Types or modules
\newcommand{\ltyp}[1]{\textcolor[HTML]{2B91AF}{#1}}

% Functions
\newcommand{\lfun}[1]{\textcolor[HTML]{0000A0}{#1}}

% Patterns
\newcommand{\lpat}[1]{\textcolor[HTML]{800080}{#1}}

% Mutable vars
\newcommand{\lvar}[1]{\textbf{\textcolor[HTML]{000000}{#1}}}

% Printf
\newcommand{\lprf}[1]{\textcolor[HTML]{2B91AF}{#1}}

% Escaped characters
\newcommand{\lesc}[1]{\textcolor[HTML]{FF0080}{#1}}

% Inactive elements
\newcommand{\inact}[1]{\textcolor[HTML]{808080}{#1}}

% Preprocessors
\newcommand{\prep}[1]{\textcolor[HTML]{0000FF}{#1}}

% fsi output
\newcommand{\fsi}[1]{\textcolor[HTML]{808080}{#1}}

% Omitted parts
\newcommand{\omi}[1]{\textcolor[HTML]{808080}{#1}}


% Overriding color and style of line numbers
\renewcommand{\theFancyVerbLine}{
\lines{\small \arabic{FancyVerbLine}:}}

\lstset{%
  backgroundcolor=\color{gray!15},
  basicstyle=\ttfamily,
  breaklines=true,
  columns=fullflexible
}

\title{Literate sample
}
\date{}

\begin{document}

\maketitle

\section*{Literate sample}



This file demonstrates how to write Markdown document with 
embedded F\# snippets that can be transformed into nice HTML 
using the \texttt{literate.fsx} script from the \href{http://fsprojects.github.io/FSharp.Formatting}{F\# Formatting
package}.


In this case, the document itself is a valid Markdown and 
you can use standard Markdown features to format the text:
\begin{itemize}
\item Here is an example of unordered list and...

\item Text formatting including \textbf{bold} and \emph{emphasis}

\end{itemize}



For more information, see the \href{http://daringfireball.net/projects/markdown}{Markdown} reference.
\subsection*{Writing F\# code}



In standard Markdown, you can include code snippets by 
writing a block indented by four spaces and the code 
snippet will be turned into a \texttt{<pre>} element. If you do 
the same using Literate F\# tool, the code is turned into
a nicely formatted F\# snippet:
\begin{Verbatim}[commandchars=\\\{\}, numbers=left]
\com{/// The Hello World of functional languages!}
\kwd{let} \kwd{rec} \lfun{factorial} \id{x} \ops{=} 
  \kwd{if} \id{x} \ops{=} \num{0} \kwd{then} \num{1} 
  \kwd{else} \id{x} \ops{*} (\lfun{factorial} (\id{x} \ops{-} \num{1}))

\kwd{let} \id{f10} \ops{=} \lfun{factorial} \num{10}

\end{Verbatim}

\subsection*{Hiding code}



If you want to include some code in the source code, 
but omit it from the output, you can use the \texttt{hide} 
command. You can also use \texttt{module=...} to specify that 
the snippet should be placed in a separate module 
(e.g. to avoid duplicate definitions).


The value will be deffined in the F\# code that is 
processed and so you can use it from other (visible) 
code and get correct tool tips:
\begin{Verbatim}[commandchars=\\\{\}, numbers=left]
\kwd{let} \id{answer} \ops{=} \ltyp{Hidden}\ops{.}\id{answer}

\end{Verbatim}

\subsection*{Including other snippets}



When writing literate programs as Markdown documents, 
you can also include snippets in other languages. 
These will not be colorized and processed as F\# 
code samples:
\begin{lstlisting}
Console.WriteLine("Hello world!");

\end{lstlisting}


This snippet is turned into a \texttt{pre} element with the
\texttt{lang} attribute set to \texttt{csharp}.




\end{document}
